\documentclass[a4paper,10pt]{article}
%\documentclass[a4paper,10pt]{scrartcl}

\usepackage[utf8]{inputenc}

\title{}
\author{}
\date{}

\pdfinfo{%
  /Title    (GDD The rise of the chicken)
  /Author   (CRamoysi)
  /Creator  (CRamoysi)
  /Producer (CRamoysi)
  /Subject  (GDD)
  /Keywords (GDD, chicken, game)
}

\begin{document}
\maketitle


\part{Présentation générale} %1

	\section{Philosophie} %1.1
		\paragraph{} Qu'y a t'il de mieux qu'une poule? Plein de poules. Qu'y a 
t'il de mieux que plein de poules? Plein de poules droguées et aggressive. 
Pourquoi etre serieux quand on peux déconner? A l'origine il s'agissait d'un 
délire pour une ludumdare. Il fallait faire un petit jeu utilisant une arme 
inhabituelle. Bien evidement la premiere idée venu à l'esprit est d'etre un 
fermier lance du grain sur des zombies qui approchent pour que ses poules les 
picorent à mort. Le tout avec une musique homemade de cotcotement de poules sur 
le theme de la chevauchée des valkyrie de Wagner. bref un bon délire qui a donné 
envie de continuer ``la liscence''. 
	\section{Questions fréquentes}%1.2
		\subsection{Qu'est-ce que ce jeu?}
			\paragraph{} Il s'agit d'un jeu de fuite de type survival. Les 
protagonistes doivent fuir une menace sans se faire attraper. Avec un style
RPG, les protagoniste doivent se deplacer sans bruit pour ne pas attirer les
poules.

		\subsection{Où le jeu se passe t'il?}
			\paragraph{} Le jeu se passe sur Terre. Different enironnement se 
	succederont tels qu'un centre commercial, un aeroport, une gare, une 
ferme, etc. Il s'agit 	d'environnement classique mais apres apocalypse. 

		\subsection{Qu'est ce que je contrôle?}
			\paragraph{} Je controle un jeune enfant sur les épaules de son 
pere.

			\subsection{De combien de personnage ai-je le contrôle?}
			\paragraph{} Je controle que le pere et l'enfant comme une seule 
entité.

		\subsection{Quel est le but du jeu?}
			\paragraph{} Le but est simple: fuir!

		\subsection{Qu'est-ce qui rend ce jeu different?}
			\paragraph{} L'univers: totalement loufoque et décalé mais avec 
		une vraie histoire. Le principe n'est pas de tirer sur tout ce qui 
bouge mais simplement de fuir, sans se faire rattraper. La gestion du bruit 
est très importante. 

\part{Mécaniques de jeu} %2
	\section{Gameplay}%2.1
		\subsection{Description générale}
			\paragraph{} Le personnage se deplace sur un plan en 2D vue de haut, 
il a ainsi possibilité de se déplacer vers le haut, le bas, la gauche et la 
droite. Une indication montrera au joueur la source de chaque bruit sous forme 
d'un cercle coloré dont le centre est la source. Le cerle sera plus ou moins 
gros et plus ou moins rouge en fonction de la distance et l'intensité du bruit. 
Il serait interessant également de gerer les obstacles qui attenueront ou 
amplifieront plus ou moins les sons.

		\subsection{Les flux du jeu}
			\paragraph{} *progression, partie typique, deroulement d'une 
partie,	défis...*

		\subsection{Les éléments du gameplay}
			\paragraph{} Le joueur pourra interagir avec son environnement tel 
qu'ouvrir des portes, pousser des objets pas trop lourd, porter un objet et le 
lancer. Il aura la possibilité de se reposer, manger, etc pour regenerer sa vie, 
sa vigueur.
			\paragraph{} Le joueur devra gerer bien evidement le bruit qu'il 
fait mais sa fatigue et sa faim. les phases de recherche de nourriture et les 
phases de repos seront importantes.
			\paragraph{} Le joueur pourra lancer des objets au loin pour y 
attirer les poules avec le bruit. Le joueurs pourra également mettre du grains 
par terre pour attirer les poules et passer discretement plus loin.

		\subsection{Données statistiques et physiques}
		
		\subsubsection{Les statistiques du personnage}
			\begin{itemize}
				\item Furtivité: Plus la furtivité est élevée et moins l'on 
genere de bruit lors des deplacements
				\item Santé: à 0 on meurt. Plus la santé est basse et moins on 
se deplace vite et plus la fatigue augmente vite.
				\item Folie: Plus la folie est élevé et plus on a de chance de 
faire du bruits (un cri, un pleur, une réaction de peur). La folie augmente 
quand 
on attire les poules, quand on fait un gros bruit d'un coup (objet qui tombe 
par exemple ou un coup de tonerre, etc). À l'inverse la folie diminue quand 
on dort, que l'on a pas eu de souci depuis un moment.
				\item Fatigue: plus la fatigue est élevée, moins on se déplace 
vite. On diminue la fatigue par le sommeil
				\item Faim: plus la faim est haute est plus on se fatigue vite 
et moins on est furtif (possibilité de gargouilli). Quand la faim est au 
maximum, on perd des pv. Pour reduire la faim, il faut manger...
			\end{itemize}

		\subsubsection{Les déplacements}
			\paragraph{} Les deplacements se font vers le haut, le bas, la 
gauche et la droite. Il faut pouvoir gerer la vitesse de deplacements (possible 
facilement avec un joystick de manette, a voir avec un clavier). Plus on se 
déplace vite et plus on fait de bruit. On sera ralenti les pieds dans 
l'eau, 	dans la boue, etc.

		\subsection{Description de l'IA}
			\paragraph{} L'IA concerne principalement les poules. Elles se 
deplaceront en fonction du bruit. Plus un bruit est fort et près et plus elles 
se precipiteront sur la source. Si un bruit n'est pas intense, il yaura une 
chance qu'elles ne se deplacent pas tout de suite, si le bruit se repete elles 
arriveront.
			\paragraph{} Si aucun bruit ne survient, elles se dirigeront vers 
de la nourriture ou bougeront de façon aléatoire le cas échéant.




		\subsection{Mécanisme d'aide pour l'apprentissage}
			\paragraph{} L'apprentissage se fera par la lecture de feuilles ou 
de livre au sol. La toute premiere aide sera noté dans le décors de façon 
évidente anoncant comment se deplacer et rammasser;

		\subsection{Durée de vie du jeu}
			\paragraph{}
		\subsection{Difficulté du jeu}
			\paragraph{} Il y aura 3 niveaux de difficultés:
				\begin{itemize}
					\item ``Poule mouillé'': Il s'agit de la difficulté 
minimal. La folie, ma faim et la fatigue ne monte pas. Les poules sont à moitié 
sourde  
					\item ``Chair de poule'': Difficulté intermediaire. La 
folie, la fatigue et la faim agissent normalement mais les poules sont à moitié 
sourde
					\item ``Poulets rotis'': La fatigue, la faim et la folie 
evoluent normalement. L'instinct prédateur des poules est à son paroxysme;
				\end{itemize}
		
		
		\subsection{}
		
\end{document}		
		
2.1.10. Modes Mono-joueur
2.1.10.1. Vue générale

Décrivez en quelques phrases le jeu mono-joueur.

2.1.10.2. Histoire

Décrivez les grandes lignes de l’histoire du jeu, en vous référant à une annexe pour les détails, en effet l’histoire d’un jeu est souvent très longue à décrire.

2.1.10.3. Les Objets de quête ou Objectifs

Décrivez les objets de quête de votre jeu ou autrement dit, les objectifs à atteindre par le joueur.

2.1.10.4. Conditions de victoire et d’échec

Quand une partie en mode mono-joueur est-elle terminée ? Comment le joueur sait-il s’il a gagné ou perdu ?

2.1.10.5. Sauver et Recharger

Possibilité de charger/sauver à n’importe quel moment, ou à certains moment bien précis, ou pas du tout.

2.1.10.6. Pause

Possibilité de mettre en pause le jeu et options éventuelles.

2.1.10.7. Personnalisation

Décrivez comment chaque joueur se verra offrir la possibilité de personnaliser son expérience de jeu. Par exemple, créer ses unités, son personnage ou son véhicule, voir son niveau, son univers de jeu.

2.1.10.8. Options de jeu

Décrivez les différentes options de jeu, mise à la disposition de l’utilisateur.

Par exemple, le réglage des commandes, du son, etc

2.1.11. Modes Multijoueurs
2.1.11.1. Vue générale

Décrivez ici le mode multi-joueur dans toute sa généralité. En particulier, ses liens avec le mode mono-joueur et les extensions qu’il propose, en dehors évidemment du fait de jouer à plusieurs.

2.1.11.2. Les Objets de quête ou Objectifs

Décrivez les objets de quête de votre jeu ou autrement dit, les objectifs à atteindre par le joueur.

2.1.11.3. Conditions de victoire et d’échec

Quand une partie en mode mono-joueur est-elle terminée ? Comment le joueur sait-il s’il a gagné ou perdu ?

2.1.11.4. Sauver et Recharger

Possibilité de charger/sauver à n’importe quel moment, ou à certains moment bien précis, ou pas du tout.

2.1.11.5. Pause

Possibilité de mettre en pause le jeu et options éventuelles.

2.1.11.6. Persistance

Votre monde est-il persistant ou lié à une partie temporaire ? Décrivez ce paragraphe de manière fonctionnelle, les solutions techniques viendront après.

2.1.11.7. Nombre de joueurs

Combien de joueurs pourront participer en même temps, avoir de comptes au total ? Une estimation de la volumétrie à traiter, utile pour dimensionner le développement.

2.1.11.8. Personnalisation

Décrivez comment chaque joueur se verra offrir la possibilité de personnaliser son expérience de jeu. Par exemple, créer ses unités, son personnage ou son véhicule, voir son niveau, son univers de jeu.

2.1.11.9. Mode écran divisé

Multi-Joueur ne rime pas forcément avec plusieurs machines, un réseau, Internet. Si votre jeu se joue à plusieurs sur un même écran, décrivez alors le mode de fonctionnement des parties. Comment l’écran est splité, comment les informations sont organisées à l’écran, etc.

2.1.11.10. Options de jeu

Décrivez les différentes options de jeu, mise à la disposition de l’utilisateur.

Par exemple, le réglage des commandes, du son, etc.

2.2. LevelDesign
2.2.1. Diagramme des niveaux

   

2.2.2. Chemin critique par niveau

Décrivez les actions nécessaires pour boucler un niveau de jeu et donc pour pouvoir continuer la partie.

2.2.3. Tableau des actifs

*Impossible de faire un tableau en bbcode*

Sous la forme d’un tableau, pour chaque niveau du jeu, vous déterminer les éléments de jeu qui seront opérationnels, actifs.

Les actifs (éléments actifs) incluent les pouvoirs, les armes, les types d’ennemis, les types objectifs, les défis, les secrets ou objets cachés, etc.

2.2.4. Eléments du décor génériques

Décrivez les éléments génériques de votre univers.

2.2.5. Eléments du décor uniques

Décrivez les éléments uniques de votre univers.

2.2.6. Blindages

Décrivez les passages obligatoires de votre jeu. Les éléments indissociables à l’avancement du joueur. Par exemple, pour finir le niveau il faut impérativement passer par une porte de couleur verte.

2.2.7. Plans par niveau

Sous la forme d’un dessin schématique, vous devez représenter chaque niveau, l’emplacement des éléments, des entités, des objets, des blindages, etc…

2.3. Moteur du jeu
2.3.1. Vue générale

Donnez une vue générale de comment votre jeu sera rendu à l’écran, mais sans aller dans les détails, les paragraphes suivants sont là pour çà.
2.3.2. Moteurs de rendu

Décrivez quel type de moteur de rendu 2D/3D sera utilisé : réaliste, cellshading ? Plutôt 2D, 3D ? Plusieurs sections pourront être rédigées si plusieurs moteurs de rendu sont utilisés.

Par exemple :

·      Moteur 1 : rendu de l’eau

·      Moteur 2 : rendu des collisions

Etc.

2.3.3. Caméras

Décrivez le fonctionnement de la caméra.

Dans ce paragraphe, non seulement vous devez décrire la vue ou les vues du jeu (vue subjective, 3ème personne, etc.) mais aussi ajouter dans des sections distinctes, certaines particularités ou spécificités de mise en scène de votre jeu si nécessaire.

manière innovante et permet de traiter tant de collisions par secondes…

2.3.4. Lumières et Ombres

Décrivez le modèle d’éclairage que vous allez implanter et détailler dans la suite.

Est-ce tu temps réel ou du précalculé ?

Est-ce que vous gérer la réflexion, la réfraction, etc. ?

2.3.5. Effets Spéciaux

Décrire les différents effets spéciaux que vous voulez inclure dans le jeu.

2.3.6. Gestion du son

Décrire la gestion du son dans le jeu vidéo.

2.3.7. Gestion physique

Décrire la gestion physique de votre jeu : détection, collisions, etc.

2.3.8. Multijoueurs

Quelle est la structure adoptée ? Par exemple, client-serveur, Peer to Peer, hébergés chez un des joueurs ou sur une ferme de serveurs, plusieurs serveurs sur une même machine ou non, etc.

Comment le jeu fonctionnera en réseau. Un mode multi-joueurs peut être limité à 2 joueurs sur une même machine ou peut impliquer un réseau local, voire des milliers de machines en réseau Peer to Peer à travers Internet.

2.3.9. Support multilangues

Décrire les langues supportées et le moteur de gestion multilangues…

3. Interface utilisateur
3.1. Vue générale

Présentez ici une vue générale de votre interface, comme d’habitude, détaillez ensuite dans des sections séparées.

3.2. Graphe des flux

Présentez ici une vue générale et schématique des différents écrans de votre interface utilisateur. Avec un outil comme VISIO (ou autre outil similaire), vous pouvez faire un schéma global qui montre les différents écrans à produire et leur enchaînement interactif.

Cela revient à traduire le système de navigation au sein de votre application.

3.3. Les procédures fonctionnelles

Décrire pour chaque écran d’interface, les actions d'utilisateur possibles et les résultats attendus.

3.4. Les objets de l’interface GUI (Graphical User Interface)

Décrire les éléments de votre interface et leurs actions : boutons, menus, icônes, etc.

Décrire également un écran type de jeu, éventuellement les écrans de jeu secondaires.

3.5. Les périphériques de contrôle

Décrire les périphériques utiles pour jouer, l’agencement des commandes de base, mais aussi des commandes combinées.

4. Editeurs (univers et entités)

Décrire les grandes lignes des éditeurs qui seront utilisés pour concevoir le jeu et peut-être fournis aux joueurs pour qu’ils puissent personnaliser divers éléments de l’univers de jeu : éditeurs de niveaux, de personnages, de véhicules, etc. Mais vous devez aussi préciser si l’utilisateur aura la possibilité de télécharger via Internet, des éléments complémentaires pour enrichir son expérience vidéo ludique.

Remarque : il est très important de réaliser que concevoir un jeu ne peut se faire sans concevoir des outils d’éditions permettant de développer l’univers. Cette partie est donc fondamentale et il est même recommandé, pour chaque éditeur, de rédiger un sous paragraphe.

5. L’Univers du jeu
5.1. Vue générale

Une vue générale du monde dans lequel se passe le jeu.

Vous devez décrire les principales caractéristiques du monde du jeu, de l’univers.

C’est ici que vous décomposez ce qui fait l’originalité du monde de votre jeu, et que vous décrivez en détails chacun de ces points clé.

Là encore, ne transformez pas ces paragraphes en dépliants publicitaires. Les points clés décrits ici ne doivent pas avoir besoin d’être exagérés, ils doivent être attractifs par eux-mêmes.

5.2. Endroits - Lieux

Décrivez d’abord les endroits clés du jeu : ceux qui sont essentiels pour le déroulement de l’histoire ou de la partie.

Et ensuite, les lieux prévus, mais pas forcément incontournables pour le bon déroulement de la partie.

5.3. Voyage

Décrivez la manière dont le joueur va progresser dans cet univers. Comment il enchaîne sa progression, les étapes, etc. Attention, ne confondez pas ce paragraphe avec le flux de jeu décrit plus haut dans la section Gameplay. Non, ici il s’agit d’expliquer le déroulement de la partie sur le plan scénaristique

5.4. Echelle

Décrivez l’échelle à laquelle vous voyez le monde de votre jeu : à travers les yeux d’un piéton, du haut d’un dragon en vol ? L’échelle conditionne beaucoup de choses dans un jeu.

5.5. Temps et Climat

Décrivez comment le déroulement du temps influera sur le jeu.

Ajoutez ici les sections pertinentes pour votre jeu. Ce peut être par exemple l’influence de la météo sur les comportements et les événements, l’existence d’un mode nuit et d’un mode jour, la prise en compte de saisons, etc.

6. Les entités de l’univers
6.1. Vue générale

Vue générale de vos entités. Les entités ce sont par exemple les personnages, ou encore les voitures si c’est un jeu de course, etc.

6.2. Les entités jouables

Décrivez les entités de façon détaillée. Il s’agit des personnages par exemple, ou des véhicules si c’est un jeu de course.

6.3. Les entités non jouables

Décrivez les entités que le joueur devra affronter ou rencontrer… Bien entendu, cela dépend du type de jeu, mais dans la plupart des jeux, les joueurs dont devoir affronter des adversaires. Il s’agit de décrire tous les aspects de ces entités adverses : leur aspect, leur caractéristiques, leurs capacités, etc. Ils peuvent être des personnages humains, des voitures, des vaisseaux, etc. Et donc il faut soigner leur description en fonction de la nature de ces entités…

6.4. Les objets

Décrire l’ensemble des objets qu’on peut récupérer et utiliser dans le jeu. Cela peut être des armes, des pièces pour booster son véhicule, etc..

Décrire également le système monétaire s’il existe un de façon précise ainsi que son fonctionnement.

6.5. Les compétences

Décrire l’ensemble des compétences de vos entités : pouvoirs, magies, etc.

Décrire également les moyens dont dispose le joueur, pour les acquérir, ainsi que leurs impacts sur l’univers ou les entités du jeu.

6.6. La création d’entités

Comment le joueur peut créer et personnaliser ses entités. Les caractéristiques disponibles et leur influence sur le jeu.

Attention, il ne s’agit pas de décrire à niveau les éditeurs disponibles, mais d’avantage ce qui est personnalisable ou pas dans le jeu

6.7. Les évolutions des différentes entités

Décrire les possibilité d’évolution des différents personnages, PNJ (personnages non jouables), de leurs compétences…

6.8. Orientation et les Déplacements

Décrire comment le joueur va appréhender l’espace du jeu, quels sont les déplacements possibles pour les différentes entités.

6.9. Interactions inter-entités et entités-univers

Décrire les interactions possibles entre les entités, mais aussi entre les entités et l’univers, l’environnement du jeu.

6.10. Moyens de locomotion

Décrire les moyens de locomotion mis à la disposition du joueur.

7. Histoire
7.1. Synopsis

En quelques lignes, décrivez le scénario ou du moins la trame, l’intrigue de votre jeu.

Mettez l’accent sur l’approche originale ou innovante que vous faites.

7.2. Découpage interactif du scénario, dialogues

Il y a différentes façons de faire. En voici une.

Votre scénario, doit être découpé en chapitres et les chapitres en sous-chapitres, etc.

Le but c’est de pouvoir articuler votre scénario ou l’histoire qui sert de fil directeur, avec l’enchaînement des niveaux, au sein de l’univers de jeu.

Par exemple, vous pouvez lister les différents niveaux de jeu et mettre en parallèle les parties du scénario qui seront traitées au sein de chaque niveau…

Idem pour les dialogues…

Puis, si vous avez des séquences vidéo (FMV ou Full Motion Video en anglais), alors préciser également les parties du scénario et dialogues qui seront incluses…

Encore une fois l’objectif final, c’est de situer le scénario et les dialogues dans le déroulement du jeu.

7.3. Références et clins d’œil

Décrire les références directes ou indirectes, ainsi que les clins d’œils éventuels.

8. Choix Artistiques : graphisme et animation
8.1. Vue générale et objectifs

Décrire en quelques lignes, les choix artistiques que vous faites et justifier ces choix.

8.2. Aspects 2D

Quel traitements graphique 2D ? Recherches graphiques…

8.3. Aspects 3D

Quel traitements graphique 3D ? Modélisations type…

8.4. Animations

Quels traitements ? Quel type d’animation ? Puis quels mouvements possibles pour les différents personnages ou entités de votre univers ?

8.5. Cinématiques

Préciser les cinématiques que vous prévoyez en insistant sur l’articulation scénaristique qu’elles vont assumer. Introduction ? Plusieurs fins de partie ? Ces cinématiques seront en images de synthèse, ou utiliserons le moteur 3D du jeu ? Pourquoi ?

8.6. Effets spéciaux

Décrire les effets spéciaux sur le plan artistique.

8.7. Interface utilisateur

Décrire l’interface utilisateur sur le plan artistique.

9. Choix Artistiques : musiques et bruitages
9.1. Vue générale et objectifs

Décrire en quelques lignes, les choix artistiques que vous faites et justifier ces choix.

9.2. Interface utilisateur et ambiance sonore

Décrire en quelques lignes le fonctionnement sur le plan sonore, de l’interface utilisateur.

Musique de fond dans les menus ou pas ? Bruitages, Voix ? Autre détails sonores ?

9.3. Musiques

Décrire en quelques lignes, les choix artistiques que vous faites et justifier ces choix.

En quoi vos choix renforcent-ils l’ambiance, l’immersion du joueur ?

Est-ce que ça colle à votre univers ?

9.4. Bruitages

Décrire en quelques lignes, les choix artistiques que vous faites et justifier ces choix.

Pour chaque entités composant votre univers, préciser les bruitages attendus, les circonstances de déclenchement, la durée, etc.

9.5. Voix

Si vous optez d’inclure des voix digitalisées pour vos personnages, précisez le type de voix souhaitée, le ton général, etc. Aussi, n’oubliez pas de tenir compte de la langue.

9.6. Effets spéciaux

Décrire en quelques lignes, les choix artistiques que vous faites et justifier ces choix.
